\section*{Conclusión}

En el primer ejercicio del sistema de Lorenz, se nota de manera inicial que al comparar el sistema discreto y continuo existe una diferencia del comportamiento en el mismo tiempo con las mismas condiciones iniciales. Al comprar las figuras \ref{img:lorenzD1} con \ref{img:lorenzD4} y \ref{img:lorenzD2} con \ref{img:lorenzD5} se observa que el sistema tiene comportamiento diferente, debido a la naturaleza caótica del sistema.\\

En el caso del filtro discreto, debido a que el integrador es uniuntegrador de paso simple de Euler existen errores en cada paso de integración lo cual hace que el sistema avance de manera muy rápida y el calculo de las ganancias de Kalman $K(k)$ puede no ser suficiente para que convergan en todo momento. Esto sucede mas que nada en los momentos donde existen cambios grandes dentro de los estados. Se puede obsrvar en la figura \ref{img:lorenzD2} cuando esta por el tiempo de 2s y 4.5s que es donde el error mostrado en la figura \ref{img:lorenzD3} tambien sufre picos.\\

En el caso continuo, el movimiento del sistema es mas suave y por lo tanto se observan cambio menos abruptos en el tiempo. El integrador en este caso es un Runge-Kutta de $4^{to}$ orden. Y el calculo de las ganancias de Kalman $K(t)$ por medio de la ecuacion de Ricatti permite que el observador converga con el sistema volviendo los errores del sistema muy pequeños tendiendo a cero con el paso del tiempo, esto se observa en la figura \ref{img:lorenzD6}.\\

En el segundo ejercicio se utilizo el sistema de Rossler para enviar una señal $\lambda$ por medio de la salida del sistema. Utilizando un identificador lineal de parametros con un estimador de mínimos cuadrados con factor de olvido se recrea la señal enviada. Esta señal se puede observar en la figura \ref{img:rossler3} en color verde y en color negro la señal recuperada $\hat{\lambda}$.

