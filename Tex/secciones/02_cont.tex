\section*{Desarrollo}
\subsection*{Ejercicio 1}
En este ejercicio, el sistema de Lorenz mostrado en la ecuación \ref{eq:lorenz} se le va a aplicar un filtro de Kalman para calcular las ganancias que permitan al observador de la ecuacion \ref{eq:lorenzobs} sincronizarse con el sistema. El sistema se encuentra alterado por ruido blanco tanto en los estados como en la salida, dichos ruidos se observan como $\epsilon_i \forall$ i = (1,2,3) y $v$.

\begin{equation}\label{eq:lorenz}
\begin{array}{c}
\dot{x}_1 = \sigma ( x_2 - x_1 ) + \epsilon_1\\
\dot{x}_2 = rx_1 - x_2 x_1x_3 + \epsilon_2\\
\dot{x}_3 = -bx_3 + x_1x_2 + \epsilon_3\\
y = x_3 + v
\end{array}
\end{equation}

\begin{equation}\label{eq:lorenzobs}
\begin{array}{c}
\dot{\hat{x}}_1 = \sigma ( \hat{x}_2 - \hat{x}_1 ) + k_1(t)e_3\\
\dot{\hat{x}}_2 = r\hat{x}_1 - \hat{x}_2 \hat{x}_1\hat{x}_3 + k(t)_2e_3\\
\dot{\hat{x}}_3 = -b\hat{x}_3 + \hat{x}_1\hat{x}_2 + k(t)_3e_3\\
\end{array}
\end{equation}



\subsubsection*{Pseudocódigo}

\begin{algorithm}[H]
	\caption{Calculate $y = x^n$}
	\begin{algorithmic}
		\REQUIRE $n \geq 0 \vee x \neq 0$
		\ENSURE $y = x^n$
		\STATE $y \leftarrow 1$
		\IF{$n < 0$}
		\STATE $X \leftarrow 1 / x$
		\STATE $N \leftarrow -n$
		\ELSE
		\STATE $X \leftarrow x$
		\STATE $N \leftarrow n$
		\ENDIF
		\WHILE{$N \neq 0$}
		\IF{$N$ is even}
		\STATE $X \leftarrow X \times X$
		\STATE $N \leftarrow N / 2$
		\ELSE[$N$ is odd]
		\STATE $y \leftarrow y \times X$
		\STATE $N \leftarrow N - 1$
		\ENDIF
		\ENDWHILE
	\end{algorithmic}
\end{algorithm}

\subsection*{Ejercicio 2}
\subsection*{Ejercicio 3}



