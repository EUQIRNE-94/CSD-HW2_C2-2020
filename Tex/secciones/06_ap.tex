\section*{Apendice\label{section:conclusion}}

\begin{Large}
Códigos Matlab
\end{Large}

	\begin{verbatim}
	% Ejercicio 1 - Tarea 2
	close all
	clear
	clc
	
	%% Sistema de Lorenz sujeto a ruido blanco en proceso y medición
	
	% Tiempo de simulación
	tf = 5;
	dt = 0.01;
	tspan = 0:dt:tf;
	n = length(tspan);
	
	%% Método Discreto - FKE general
	
	% Prealocacion de variables
	x = zeros(3,n);                                                             % Sistema
	xg = zeros(3,n);                                                            % Observador del sistema
	y = zeros(1,n);                                                             % Salida del sistema
	xg_k = zeros(3,n);                                                          % x(k+1|k)
	K = zeros(3,n);                                                             % Ganancias
	P = zeros(3,3,n);                                                           % Error de covarianza
	Q = zeros(3,3,n);                                                           % Matriz de covarianza del sistema
	R = zeros(1,n);                                                             % Covarianza de la salida
	eps = zeros(3,n);                                                           % Ruidos del proceso
	v = zeros(1,n);                                                             % Ruido de lectura
	e1 = zeros(1,n);                                                            % Error de variable medida
	
	% Ruido del proceso y lectura
	mu = 0;                                                                     % Centro del ruido
	sig = 0.01;                                                                 % Varianza
	% Condiciones iniciales
	x(:,1) = rand(3,1);
	xg_k(:,1) = x(:,1);
	P(:,:,1) = rand(3);
	
	% Parametros del sistema
	sigma = 10;
	r = 28;
	b = 8/3;
	
	% Matrices lineales
	A = [-sigma,sigma,0; r, -1, 0; 0,0,-b];
	
	for k = 1:n
	% Ruido del sistema y salida
	eps(:,k) = mu + sig*randn(1,3);
	v(1,k) = mu + sig*randn(1,1);
	% Calcular Sistema con ruido
	B = [0;-x(1,k)*x(3,k);x(1,k)*x(2,k)];
	x(:,k+1) = x(:,k) + ( A*x(:,k) + B )*dt + eps(:,k);
	y(1,k) = x(1,k) + v(1,k);
	
	% Obtener el estimado a posteriori del estado
	Bg = [0;-xg_k(1,k)*xg_k(3,k);xg_k(1,k)*xg_k(2,k)];
	xg_k(:,k+1) = xg_k(:,k) + ( A*xg_k(:,k) + Bg )*dt;
	
	% Obtener error de covarianza estimado
	F = [ 1-sigma*dt,sigma*dt,0 ; (r-x(3,k))*dt,1-dt,-x(1,k)*dt ; x(2,k)*dt,x(1,k)*dt,1-b*dt ];
	c_e1 = covarianza(eps(1,1:k),eps(1,1:k));
	c_e2 = covarianza(eps(2,1:k),eps(2,1:k));
	c_e3 = covarianza(eps(3,1:k),eps(3,1:k));
	%   Q(:,:,k) = [c_e1,0,0;0,c_e2,0;0,0,c_e3];
	Q(:,:,k) = [sig*sig,0,0;0,sig*sig,0;0,0,sig*sig];
	P_k = F*P(:,:,k)*F' + Q(:,:,k);
	
	% Obtener ganancias de Kalman
	H = [1,0,0];
	%   R(1,k) = covarianza(v(1,1:k),v(1,1:k));
	R(1,k) = sig*sig;
	K(:,k) = P_k*H'*inv( H*P_k*H' + R(1,k) );
	
	% Generar copia del sistema con ganancias de Kalman
	e1(1,k) = y(1,k) - xg(1,k);
	xg(:,k+1) = xg_k(:,k) + K(:,k)*(e1(1,k))*dt;
	
	% Actualizar error de covarianza
	P(:,:,k+1) = (eye(3) - K(:,k)*H)*P_k;  
	end
	
	% Errores
	e2 = x(2,:) - xg(2,:);
	e3 = x(3,:) - xg(3,:);
	
	%% Metodo continuo - Paper
	tspan_c = [0 tf];
	
	x0_L = [x(1,1),x(2,1),x(3,1),0,0,0,.1,.1,.1,.1,.1,.1,.1,.1,.1];
	[t_L,x_L] = ode45(@LORENZ,tspan_c,x0_L);
	
	%% Figuras
	figure(1)
	set(gcf, 'Position', get(0, 'Screensize'));
	% Estado 1
	subplot(3,1,1)
	plot(tspan,x(1,1:n),'k','linewidth',4);hold on;grid on
	plot(tspan,xg(1,1:n),'g','linewidth',2)
	title('Estado $x_1$','interpreter','latex','fontsize',30)
	xlabel({'Tiempo $t$'},'Interpreter','latex','fontsize',20)
	ylabel({'$x_1$'},'Interpreter','latex','fontsize',20)
	legend({'Sistema','Observador',},'fontsize',16)
	
	% Estado 2
	subplot(3,1,2)
	plot(tspan,x(2,1:n),'k','linewidth',4);hold on; grid on
	plot(tspan,xg(2,1:n),'g','linewidth',2)
	title('Estado $x_2$','interpreter','latex','fontsize',30)
	xlabel({'Tiempo $t$'},'Interpreter','latex','fontsize',20)
	ylabel({'$x_2$'},'Interpreter','latex','fontsize',20)
	legend({'Sistema','Observador',},'fontsize',16)
	
	% Estado 3
	subplot(3,1,3)
	plot(tspan,x(3,1:n),'k','linewidth',4);hold on; grid on
	plot(tspan,xg(3,1:n),'g','linewidth',2)
	title('Estado $x_3$','interpreter','latex','fontsize',30)
	xlabel({'Tiempo $t$'},'Interpreter','latex','fontsize',20)
	ylabel({'$x_3$'},'Interpreter','latex','fontsize',20)
	legend({'Sistema','Observador',},'fontsize',16)
	
	% set(gca,'LooseInset',get(gca,'TightInset'));
	% saveas(gcf,'E1_Estados_Disc.png')
	
	% Errores
	figure(2)
	set(gcf, 'Position', get(0, 'Screensize'));
	plot(tspan,e1,'r','linewidth',2); hold on; grid on
	plot(tspan,e2(1:n),'g','linewidth',2)
	plot(tspan,e3(1:n),'b','linewidth',2)
	title('Errores','interpreter','latex','fontsize',30)
	xlabel({'Tiempo $t$'},'Interpreter','latex','fontsize',20)
	ylabel({'$e_i$'},'Interpreter','latex','fontsize',20)
	legend({'e_1','e_2','e_3'},'fontsize',16)
	
	% set(gca,'LooseInset',get(gca,'TightInset'));
	% saveas(gcf,'E1_Errores_Disc.png')
	
	% Retrato Fase
	figure(3)
	set(gcf, 'Position', get(0, 'Screensize'));
	plot3(x(1,:),x(2,:),x(3,:),'k','linewidth',1); hold on; grid on
	plot3(xg(1,:),xg(2,:),xg(3,:),'g','linewidth',1)
	title('Retrato Fase','fontsize',30)
	xlabel({'$x_1$'},'Interpreter','latex','fontsize',20)
	ylabel({'$x_2$'},'Interpreter','latex','fontsize',20)
	zlabel({'$x_3$'},'Interpreter','latex','fontsize',20)
	legend({'Sistema','Observador',},'fontsize',16)
	
	% set(gca,'LooseInset',get(gca,'TightInset'));
	% saveas(gcf,'E1_RetratoFase_Disc.png')
	
	%% Figuras sistema continuo
	figure(4)
	set(gcf, 'Position', get(0, 'Screensize'));
	% Estado 1
	subplot(3,1,1)
	plot(t_L,x_L(:,1),'k','linewidth',4);hold on;grid on
	plot(t_L,x_L(:,4),'g','linewidth',2);
	title('Estado $x_1$','interpreter','latex','fontsize',30)
	xlabel({'Tiempo $t$'},'Interpreter','latex','fontsize',20)
	ylabel({'$x_1$'},'Interpreter','latex','fontsize',20)
	legend({'Sistema','Observador',},'fontsize',16)
	
	% Estado 2
	subplot(3,1,2)
	plot(t_L,x_L(:,2),'k','linewidth',4);hold on;grid on
	plot(t_L,x_L(:,5),'g','linewidth',2);
	title('Estado $x_2$','interpreter','latex','fontsize',30)
	xlabel({'Tiempo $t$'},'Interpreter','latex','fontsize',20)
	ylabel({'$x_2$'},'Interpreter','latex','fontsize',20)
	legend({'Sistema','Observador',},'fontsize',16)
	
	% Estado 3
	subplot(3,1,3)
	plot(t_L,x_L(:,3),'k','linewidth',4);hold on;grid on
	plot(t_L,x_L(:,6),'g','linewidth',2);
	title('Estado $x_3$','interpreter','latex','fontsize',30)
	xlabel({'Tiempo $t$'},'Interpreter','latex','fontsize',20)
	ylabel({'$x_3$'},'Interpreter','latex','fontsize',20)
	legend({'Sistema','Observador',},'fontsize',16)
	
	% set(gca,'LooseInset',get(gca,'TightInset'));
	% saveas(gcf,'E1_Estados_Cont.png')
	
	% Errores
	e1L = x_L(:,1) - x_L(:,4);
	e2L = x_L(:,2) - x_L(:,5);
	e3L = x_L(:,3) - x_L(:,6);
	
	figure(5)
	set(gcf, 'Position', get(0, 'Screensize'));
	plot(t_L,e1L,'r','linewidth',2); hold on; grid on
	plot(t_L,e2L,'g','linewidth',2)
	plot(t_L,e3L,'b','linewidth',2)
	title('Errores','interpreter','latex','fontsize',30)
	xlabel({'Tiempo $t$'},'Interpreter','latex','fontsize',20)
	ylabel({'$e_i$'},'Interpreter','latex','fontsize',20)
	legend({'e_1','e_2','e_3'},'fontsize',16)
	
	% set(gca,'LooseInset',get(gca,'TightInset'));
	% saveas(gcf,'E1_Errores_Cont.png')
	
	% Retrato Fase
	figure(6)
	set(gcf, 'Position', get(0, 'Screensize'));
	plot3(x_L(:,1),x_L(:,2),x_L(:,3),'b','linewidth',1); hold on; grid on
	plot3(x_L(:,4),x_L(:,5),x_L(:,6),'g','linewidth',1);
	title('Retrato Fase','fontsize',30)
	xlabel({'$x_1$'},'Interpreter','latex','fontsize',20)
	ylabel({'$x_2$'},'Interpreter','latex','fontsize',20)
	zlabel({'$x_3$'},'Interpreter','latex','fontsize',20)
	legend({'Sistema','Observador',},'fontsize',16)
	
	% set(gca,'LooseInset',get(gca,'TightInset'));
	% saveas(gcf,'E1_RetratoFase_Cont.png')
	
	%% Funciones
	% FKE
	function c = covarianza(x,y)
	n_x = length(x);
	n_y = length(y);
	
	if n_x == n_y
	% Promedio de variables
	x_b = mean(x);
	y_b = mean(y);
	% Calculo de covarianza
	c = sum( (x - x_b).*(y - y_b)  )/n_x;
	end
	
	end
	
	% Paper
	function dx = LORENZ(~,x)
	
	sigma = 10;
	r = 28;
	b = 8/3;
	sig = 0.01;
	
	x1 = x(1);
	x2 = x(2);
	x3 = x(3);
	
	E1 = sig*randn();
	E2 = sig*randn();
	E3 = sig*randn();
	
	dx(1) = sigma*(x2 - x1) + E1;
	dx(2) = r*x1 - x2 - x1*x3 + E2;
	dx(3) = -b*x3 + x1*x2 + E3;
	
	v = sig*randn();
	y = x1 + v;
	
	% Filtro Kalman
	p11 = x(7);
	p12 = x(8);
	p13 = x(9);
	p21 = x(10);
	p22 = x(11);
	p23 = x(12);
	p31 = x(13);
	p32 = x(14);
	p33 = x(15);
	
	P = [p11,p12,p13;p21,p22,p23;p31,p32,p33];
	F = [-sigma,sigma,0 ; r-x3,1,-x1 ; x2,x1,-b];
	H = [1,0,0];
	Q = [sig^2,0,0;0,sig^2,0;0,0,sig^2];
	R = sig^2;
	
	Pp = F*P + P*F' - P*H'*(1/R)*H*P + Q;
	
	dx(7) = Pp(1,1);
	dx(8) = Pp(1,2);
	dx(9) = Pp(1,3);
	dx(10) = Pp(2,1);
	dx(11) = Pp(2,2);
	dx(12) = Pp(2,3);
	dx(13) = Pp(3,1);
	dx(14) = Pp(3,2);
	dx(15) = Pp(3,3);
	
	K = P*[1;0;0]*(1/R);
	
	xg1 = x(4);
	xg2 = x(5);
	xg3 = x(6);
	
	e1 = y - xg1;
	
	dx(4) = sigma*(xg2 - xg1) + K(1)*e1;
	dx(5) = r*xg1 - xg2 - xg1*xg3 + K(2)*e1;
	dx(6) = -b*xg3 + xg1*xg2 + K(3)*e1;
	
	dx = dx';
	end
	\end{verbatim}

	\begin{verbatim}
	% Ejercicio 2 - Tarea 2
	close all
	clear
	clc
	
	%% Sistema Transmisor de Rossler con método de identificcion de parametros lineal
	
	tspan = [0 200];
	dt = 0.1;
	
	n = (tspan(2)/dt) +1;
	tL = zeros(1,n);
	lambda = zeros(1,n);
	for i = 1:n
	tL(i) = (i-1)*dt;
	lambda(i) = LAMBDA(tL(i));
	end
	
	% Sistema de Rossler con cambio de variables
	x0_R = [1,1,log(1),0,0,0,0,0,0,0,0,0,0,rand()];
	[t,x] = ode45(@(t,x)lambdaRossler(t,x),tspan,x0_R);
	x(:,3) = exp(x(:,3));
	
	% Figuras Rossler con cambio de variables
	figure(1)
	set(gcf, 'Position', get(0, 'Screensize'));
	subplot(3,1,1)
	plot(t,x(:,1),'k','linewidth',2);
	title('$x_1$','Interpreter','latex','fontsize',30)
	xlabel('Tiempo [t]')
	ylabel('Uds')
	
	subplot(3,1,2)
	plot(t,x(:,2),'k','linewidth',2);
	title('$x_2$','Interpreter','latex','fontsize',30)
	xlabel('Tiempo [t]')
	ylabel('Uds')
	
	subplot(3,1,3)
	plot(t,x(:,3),'k','linewidth',2);
	title({'$Senal - x_3$'},'Interpreter','latex','fontsize',30)
	xlabel('Tiempo [t]')
	ylabel('Uds')
	
	set(gca,'LooseInset',get(gca,'TightInset'));
	saveas(gcf,'E2_Estados.png')
	
	% Retrato Fase
	figure(2)
	set(gcf, 'Position', get(0, 'Screensize'));
	plot3(x(:,1),x(:,2),x(:,3),'k','linewidth',1)
	grid on
	title('Retrato Fase Sistema','fontsize',30)
	xlabel({'$x_1$'},'Interpreter','latex','fontsize',20)
	ylabel({'$x_2$'},'Interpreter','latex','fontsize',20)
	zlabel({'$x_3$'},'Interpreter','latex','fontsize',20)
	set(gca,'LooseInset',get(gca,'TightInset'));
	saveas(gcf,'E2_RetratoFase.png')
	
	% Lambda
	figure(3)
	set(gcf, 'Position', get(0, 'Screensize'));
	plot(tL,lambda,'g','linewidth',2); hold on; grid on
	plot(t,x(:,13),'k','linewidth',2)
	title('$\lambda \ y \ \hat{\lambda}$','interpreter','latex','fontsize',30)
	xlabel({'Tiempo $t$'},'Interpreter','latex','fontsize',20)
	ylabel({'$\lambda \ / \ \hat{\lambda}$'},'Interpreter','latex','fontsize',20)
	legend({'$\lambda$','$\hat{\lambda}$'},'interpreter','latex','fontsize',16)
	set(gca,'LooseInset',get(gca,'TightInset'));
	saveas(gcf,'E2_Lambda.png')
	
	% Error de Lambda
	lambdaI = interp1(t,x(:,13),tL);
	eL = lambda - lambdaI;
	figure(4)
	set(gcf, 'Position', get(0, 'Screensize'));
	plot(tL,eL,'g','linewidth',2); grid on
	title('Error de \lambda','fontsize',30)
	xlabel({'Tiempo $t$'},'Interpreter','latex','fontsize',20)
	ylabel({'$error$'},'Interpreter','latex','fontsize',20)
	set(gca,'LooseInset',get(gca,'TightInset'));
	saveas(gcf,'E2_ErrorLambda.png')
	
	
	%% Funciones
	% Obtener lambda de la salida del sistema
	function dx = lambdaRossler(t,x)
	%% Sistema
	x1 = x(1);
	x2 = x(2);
	x3 = x(3);
	lambda = LAMBDA(t);
	
	dx(1) = -x2 - exp(x3);
	dx(2) = x1 + lambda*x2;
	dx(3) = x1 + 2*exp(-x3) - 4;
	y = x3;
	
	%% Reconstruccion de Lambda
	x3p = exp(x3);
	u1 = -x3p;
	u2  = (2/x3p) - 4;
	
	k0 = 512;
	k1 = 192;
	k2 = 24;
	v = 800;
	gamma = 0.002;
	
	w01 = x(4);
	w02 = x(5);
	w03 = x(6);
	w11 = x(7);
	w12 = x(8);
	w13 = x(9);
	w21 = x(10);
	w22 = x(11);
	w23 = x(12);
	lambdag = x(13);
	p = x(14);
	
	dx(4) = w02;
	dx(5) = w03;
	dx(6) = -k0*w01 -k1*w02 - k2*w03 + y;
	dx(7) = w12;
	dx(8) = w13;
	dx(9) = -k0*w11 -k1*w12 - k2*w13 + u1;
	dx(10) = w22;
	dx(11) = w23;
	dx(12) = -k0*w21 -k1*w22 - k2*w23 + u2;
	
	phi0 = k0*w01 + (k1 - 1)*w02 + k2*w03 + w12 + w21 + w23;
	phi1 = w03 - w11 - w22;
	
	yg = phi0 + lambdag*phi1;
	
	dx(13) = -v*phi1*p*(yg - y);
	dx(14) = -v*( (phi1^2)*(p^2) - gamma*p );
	
	dx = dx';
	end
	
	function y = LAMBDA(t)
	if t < 100
	y = 0.3;
	else
	y = 0.3 + 0.2*sin((pi/25)*t);
	end
	end
	\end{verbatim}
	

	\begin{verbatim}
	% Ejercicio 3 - Tarea 2
	close all
	clear
	clc
	
	%% Sistema Transmisor de Rossler con observador adaptable para comunicaciones
	tspan = [0 2];
	dt = 0.01;
	tspan1 = 0:dt:2;
	
	n = (tspan(2)/dt) +1;
	tL = zeros(1,n);
	lambda = zeros(1,n);
	for i = 1:n
	tL(i) = (i-1)*dt;
	lambda(i) = LAMBDA(tL(i));
	end
	
	% Sistema de Rossler con cambio de variables
	x0_R = [.1,.1,log(.1),.1,.1,log(.1),.1,.1,0,0,0,0,0,0,0.1];
	[t,x] = ode45(@(t,x)lambdaRossler(t,x),tspan1,x0_R);
	x(:,3) = exp(x(:,3));
	
	% Figuras Rossler con cambio de variables
	% figure(1)
	% set(gcf, 'Position', get(0, 'Screensize'));
	% subplot(3,1,1)
	% plot(t,x(:,1),'k','linewidth',2);hold on
	% plot(t,x(:,4),'r','linewidth',2);
	% plot(t,x(:,7),'b','linewidth',2);
	% title('$x_1$','Interpreter','latex','fontsize',30)
	% xlabel('Tiempo [t]')
	% ylabel('Uds')
	% legend({'$\xi$','Z','$\xi_{filtrada}$'},'interpreter','latex','Fontsize',16)
	% 
	% subplot(3,1,2)
	% plot(t,x(:,2),'k','linewidth',2);hold on
	% plot(t,x(:,5),'r','linewidth',2);
	% plot(t,x(:,8),'b','linewidth',2);
	% title('$x_2$','Interpreter','latex','fontsize',30)
	% xlabel('Tiempo [t]')
	% ylabel('Uds')
	% legend({'$\xi$','Z','$\xi_{filtrada}$'},'interpreter','latex','Fontsize',16)
	% 
	% subplot(3,1,3)
	% plot(t,x(:,3),'k','linewidth',2);hold on
	% plot(t,x(:,6),'r','linewidth',2);
	% title({'$Senal - x_3$'},'Interpreter','latex','fontsize',30)
	% xlabel('Tiempo [t]')
	% ylabel('Uds')
	% legend({'$\xi$','Z'},'interpreter','latex','Fontsize',16)
	% 
	% % Retrato Fase
	% figure(2)
	% set(gcf, 'Position', get(0, 'Screensize'));
	% plot3(x(:,1),x(:,2),x(:,3),'k','linewidth',1);hold on
	% plot3(x(:,4),x(:,5),x(:,6),'r','linewidth',1)
	% grid on
	% title('Retrato Fase Sistema X y Z','fontsize',30)
	% xlabel({'$x_1$'},'Interpreter','latex','fontsize',20)
	% ylabel({'$x_2$'},'Interpreter','latex','fontsize',20)
	% zlabel({'$x_3$'},'Interpreter','latex','fontsize',20)
	
	figure(3)
	set(gcf, 'Position', get(0, 'Screensize'));
	subplot(3,1,1)
	plot(t,x(:,9),'k','linewidth',2);hold on
	plot(t,x(:,12),'r','linewidth',1)
	title({'$\eta_1 \ / \ \hat{\eta}_1$'},'Interpreter','latex','fontsize',30)
	xlabel('Tiempo [t]')
	ylabel('Uds')
	legend({'$\eta_1$','$\hat{\eta}_1$'},'interpreter','latex','Fontsize',16)
	
	subplot(3,1,2)
	plot(t,x(:,10),'k','linewidth',2);hold on
	plot(t,x(:,13),'r','linewidth',1);
	title({'$\eta_1 \ / \ \hat{\eta}_1$'},'Interpreter','latex','fontsize',30)
	xlabel('Tiempo [t]')
	ylabel('Uds')
	legend({'$\eta_1$','$\hat{\eta}_1$'},'interpreter','latex','Fontsize',16)
	
	
	subplot(3,1,3)
	plot(t,x(:,11),'k','linewidth',2);hold on
	plot(t,x(:,14),'r','linewidth',1);
	title({'$y \ / \ \hat{y}$'},'Interpreter','latex','fontsize',30)
	xlabel('Tiempo [t]')
	ylabel('Uds')
	legend({'$y$','$\hat{y}$'},'interpreter','latex','Fontsize',16)
	% set(gca,'LooseInset',get(gca,'TightInset'));
	% saveas(gcf,'E3_Estados.png')
	
	% Lambda
	figure(4)
	set(gcf, 'Position', get(0, 'Screensize'));
	plot(tL,lambda,'g','linewidth',2); hold on; grid on
	plot(t,x(:,15),'k','linewidth',2)
	title('$\lambda \ y \ \hat{\lambda}$','interpreter','latex','fontsize',30)
	xlabel({'Tiempo $t$'},'Interpreter','latex','fontsize',20)
	ylabel({'$\lambda \ / \ \hat{\lambda}$'},'Interpreter','latex','fontsize',20)
	legend({'$\lambda$','$\hat{\lambda}$'},'interpreter','latex','fontsize',16)
	% set(gca,'LooseInset',get(gca,'TightInset'));
	% saveas(gcf,'E3_Lambda.png')
	
	% % Error de Lambda
	% set(gcf, 'Position', get(0, 'Screensize'));
	% lambdaI = interp1(t,x(:,15),tL);
	% eL = lambda - lambdaI;
	% figure(5)
	% set(gcf, 'Position', get(0, 'Screensize'));
	% plot(tL,eL,'g','linewidth',2); grid on
	% title('Error de \lambda','fontsize',30)
	% xlabel({'Tiempo $t$'},'Interpreter','latex','fontsize',20)
	% ylabel({'$error$'},'Interpreter','latex','fontsize',20)
	% % set(gca,'LooseInset',get(gca,'TightInset'));
	% % saveas(gcf,'E3_ErrorLambda.png')
	
	%% Funciones
	% Obtener lambda de la salida del sistema
	function dx = lambdaRossler(t,x)
	%% Sistema
	E1 = x(1);
	E2 = x(2);
	E3 = x(3);
	lambda = LAMBDA(t);
	
	% Sistema con cambio de variable
	dx(1) = -E2 - exp(E3);
	dx(2) = E1 + lambda*E2;
	dx(3) = E1 + 2*exp(-E3) - 4;
	y = E3;
	
	% Cambio de coordenadas a Z
	z1 = x(4);
	z2 = x(5);
	z3 = x(6);
	
	dx(4) = 4*exp(-y) - 2 + lambda*exp(y);
	dx(5) = z1 - z3 - exp(y) + lambda*(2 - 4*exp(-y));
	dx(6) = z2 + 4*exp(-y) - 2 + lambda*y;
	yz = z3;
	
	% Transformacion Filtrada
	k1 = 1;
	k2 = 1;
	E1f = x(7);
	E2f = x(8);
	
	dx(7) = k1*E2f + k1*yz + exp(yz);
	dx(8) = E1f + k2*E2f + k2*yz - 4*exp(yz) + 2;
	
	% Sistema con nuevas coordenadas
	eta1 = x(9);
	eta2 = x(10);
	y_nc = x(11); %
	
	dx(9) = k1*eta2 - k1*k2*y_nc + (k1 + 1)*(4*exp(-y_nc) - 2);
	dx(10) = eta1 + k2*eta2 - ( k1 + k2*k2 + 1 )*y_nc + k2*(4*exp(-y_nc) - 2) - exp(y_nc);
	dx(11) = eta2 + k2*y_nc + 4*exp(-y_nc) - 2 + lambda*(E2f + y_nc);  %
	
	% Observador adaptable
	l1 = .01;
	l2 = .01;
	l3 = .01;
	gamma = 1;
	eta1_g = x(12);
	eta2_g = x(13);
	y_ncg = x(14);
	teta = x(15);
	
	dx(12) = k1*eta2_g - k1*k2*y_ncg + (k1 + 1)*(4*exp(-y_nc) - 2) + l1*(y_ncg - y_nc);
	dx(13) = eta1_g + k2*eta2_g - ( k1 + k2*k2 + 1 )*y_ncg + k2*(4*exp(-y_nc) - 2) - exp(y_nc) + l2*(y_ncg - y_nc);
	dx(14) = eta2_g + k2*y_ncg + 4*exp(-y_nc) - 2 + teta*(E2f + y_nc) + l3*(y_ncg - y_nc);
	
	dx(15) = -gamma*(E2f + y_nc)*(y_ncg - y_nc);
	
	dx = dx';
	end
	
	function y = LAMBDA(t)
	if t < 25
	y = 0.3;
	else
	y = 0.3 + 0.2*sin((pi/25)*t - 25);
	end
	end
	\end{verbatim}
	